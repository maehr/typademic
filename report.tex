\documentclass[11pt,ngerman,a4paper,]{scrartcl}
\usepackage{lmodern}
\usepackage{setspace}
\setstretch{1.5}
\usepackage{amssymb,amsmath}
\usepackage{ifxetex,ifluatex}
\usepackage{fixltx2e} % provides \textsubscript
\ifnum 0\ifxetex 1\fi\ifluatex 1\fi=0 % if pdftex
  \usepackage[T1]{fontenc}
  \usepackage[utf8]{inputenc}
\else % if luatex or xelatex
  \ifxetex
    \usepackage{mathspec}
  \else
    \usepackage{fontspec}
  \fi
  \defaultfontfeatures{Ligatures=TeX,Scale=MatchLowercase}
\fi
% use upquote if available, for straight quotes in verbatim environments
\IfFileExists{upquote.sty}{\usepackage{upquote}}{}
% use microtype if available
\IfFileExists{microtype.sty}{%
\usepackage{microtype}
\UseMicrotypeSet[protrusion]{basicmath} % disable protrusion for tt fonts
}{}
\usepackage[lmargin=3cm,rmargin=3cm,tmargin=3cm,bmargin=3cm]{geometry}
\usepackage{hyperref}
\PassOptionsToPackage{usenames,dvipsnames}{color} % color is loaded by hyperref
\hypersetup{unicode=true,
            colorlinks=true,
            linkcolor=black,
            citecolor=black,
            urlcolor=black,
            breaklinks=true}
\urlstyle{same}  % don't use monospace font for urls
\ifnum 0\ifxetex 1\fi\ifluatex 1\fi=0 % if pdftex
  \usepackage[shorthands=off,main=ngerman]{babel}
\else
  \usepackage{polyglossia}
  \setmainlanguage[]{german}
\fi
\IfFileExists{parskip.sty}{%
\usepackage{parskip}
}{% else
\setlength{\parindent}{0pt}
\setlength{\parskip}{6pt plus 2pt minus 1pt}
}
\setlength{\emergencystretch}{3em}  % prevent overfull lines
\providecommand{\tightlist}{%
  \setlength{\itemsep}{0pt}\setlength{\parskip}{0pt}}
\setcounter{secnumdepth}{0}
% Redefines (sub)paragraphs to behave more like sections
\ifx\paragraph\undefined\else
\let\oldparagraph\paragraph
\renewcommand{\paragraph}[1]{\oldparagraph{#1}\mbox{}}
\fi
\ifx\subparagraph\undefined\else
\let\oldsubparagraph\subparagraph
\renewcommand{\subparagraph}[1]{\oldsubparagraph{#1}\mbox{}}
\fi

%%% Use protect on footnotes to avoid problems with footnotes in titles
\let\rmarkdownfootnote\footnote%
\def\footnote{\protect\rmarkdownfootnote}

%%% Change title format to be more compact
\usepackage{titling}

% Create subtitle command for use in maketitle
% \newcommand{\subtitle}[1]{
%   \posttitle{
%     \begin{center}\large#1\end{center}
%     }
% }

\setlength{\droptitle}{-2em}
  \title{}
  \pretitle{\vspace{\droptitle}}
  \posttitle{}
  \author{}
  \preauthor{}\postauthor{}
  \date{}
  \predate{}\postdate{}

\flushbottom
\clubpenalty10000
\widowpenalty10000
\displaywidowpenalty=10000
\interfootnotelinepenalty=10000
\setkomafont{sectioning}{}

\begin{document}

{
\hypersetup{linkcolor=black}
\setcounter{tocdepth}{3}
\tableofcontents
}
\listoftables
\listoffigures
\hypertarget{zusammenfassung}{%
\section{Zusammenfassung}\label{zusammenfassung}}

Das Dissertationsprojekt von Moritz Mähr geht der Entstehung des
Zentralen Ausländerregisters (ZAR) in den 1970er und 1980er Jahren nach.
Das Teilprojekt untersucht den Wandel der Erwartungen an die
Soziotechnik „Registrieren`` im Kontext computergestützter Prozesse in
der Verwaltung. Durch die standardisierte Erhebung am Bildschirm und die
zentrale Speicherung auf Magnetband wird die ausländische Bevölkerung
einer elektronischen Datenbank zugeführt, die sie nicht nur
quantifizierbar, sondern auch politisch steuer- und polizeilich
kontrollierbar macht. Welche wirtschafts- und sicherheitspolitischen
Kontrollphantasien sind mit der Einrichtung des ZAR verbunden? Wie
verändern sich Zuständigkeiten zwischen Bundesämtern durch die zentral,
mit Computer erfassten Personendaten der ausländischen Wohnbevölkerung.

\emph{Schlüsselworte: Personenregister, Computergeschichte,
Verwaltungsgeschichte,}

\hypertarget{forschungsplan}{%
\section{Forschungsplan}\label{forschungsplan}}

\hypertarget{forschungsstand}{%
\subsection{Forschungsstand}\label{forschungsstand}}

\hypertarget{stand-der-eigenen-forschung}{%
\subsection{Stand der eigenen
Forschung}\label{stand-der-eigenen-forschung}}

{[}Beschreiben Sie Ihre eigenen Forschungsarbeiten im Fachgebiet des
Projektes, deren bisheriger Ergebnisse sowie die Relevanz dieser
Vorarbeiten für das aktuelle Projekt. Falls Ihr Projekt auf einem vom
SNF unterstützten Vorprojekt aufbaut, berichten Sie über die
durchgeführten Arbeiten und die im Vorprojekt erreichten Resultate.{]}

\hypertarget{detaillierter-forschungsplan}{%
\subsection{Detaillierter
Forschungsplan}\label{detaillierter-forschungsplan}}

{[}Geben Sie aufbauend auf Angaben unter Punkten 2.1 und 2.2 an, welche
Forschungsansätze Sie verfolgen und welche konkreten Ziele Sie in der
Gesuchsperiode zu erreichen gedenken. - Beschreiben Sie die konkreten
Untersuchungen bzw. Experimente, die zur Erreichung dieser Ziele zur
Anwendung kommen. Beurteilen Sie die Risiken des Vorgehens und schlagen
Sie falls nötig Alternativen vor. - Charakterisieren Sie bestehende
Quellen und Datensets und beschreiben Sie die Datensammlungsstrategie
und allfällige Alternativstrategien. - Beschreiben Sie die Rolle jedes
einzelnen Mitglieds des Forschungsteams (inkl. Gesuchstellende,
Mitarbeitende, Projektpartner, und Kollaborationen). Der Umfang und
Detailgrad der Angaben sollen den Experten erlauben, die Angemessenheit
des methodischen Vorgehens und die Machbarkeit Ihres Forschungsvorhabens
zu beurteilen. Bitte beziehen Sie sich im Budget, welches Sie über mySNF
eingeben, auf die hier beschriebenen Arbeiten.{]}

\hypertarget{zeitplan-und-etappenziel}{%
\subsection{Zeitplan und Etappenziel}\label{zeitplan-und-etappenziel}}

Erstellen Sie einen Zeitplan mit den wichtigsten Etappenzielen
(Milestones).

\hypertarget{bedeutsamkeit-der-forschungsarbeit}{%
\subsection{Bedeutsamkeit der
Forschungsarbeit}\label{bedeutsamkeit-der-forschungsarbeit}}

\hypertarget{wissenschaftliche-bedeutsamkeit}{%
\subsubsection{Wissenschaftliche
Bedeutsamkeit}\label{wissenschaftliche-bedeutsamkeit}}

Beschreiben Sie die möglichen Auswirkungen dieses Projektes für das
Fachgebiet und die Wissenschaft im Allgemeinen (Forschung und Ausbildung
bzw. Lehre). Geben Sie an, in welcher Form Sie die Forschungsergebnisse
publizieren möchten (Artikel in wissenschaftlichen Zeitschriften,
Monographien, Tagungsberichten usw.).

\hypertarget{ausserwissenschaftliche-bedeutsamkeit-broader-impact}{%
\subsubsection{Ausserwissenschaftliche Bedeutsamkeit (broader
impact)}\label{ausserwissenschaftliche-bedeutsamkeit-broader-impact}}

Falls Sie Ihr Gesuch als anwendungsorientiert deklariert haben,
beschreiben Sie die ausserwissenschaftliche Bedeutsamkeit des
vorgeschlagenen Projektes. Berücksichtigen Sie dabei folgende Punkte: -
Definieren Sie den Forschungsbedarf aus Sicht der Praxis/Industrie.
Welche Lücken im Kenntnisstand bestehen, welche Neuentwicklungen oder
Verbesserungen werden erwartet? - Inwieweit können die erwarteten
Forschungsergebnisse in die Praxis umgesetzt bzw. angewendet werden? -
In welchen ausserwissenschaftlichen Bereichen kann eine Umsetzung der
Forschungsergebnisse voraussichtlich welche Veränderungen bewirken?

\hypertarget{literatur--und-quellenverzeichnis}{%
\section{Literatur- und
Quellenverzeichnis}\label{literatur--und-quellenverzeichnis}}


\end{document}
